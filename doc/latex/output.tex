\PassOptionsToPackage{unicode=true}{hyperref} % options for packages loaded elsewhere
\PassOptionsToPackage{hyphens}{url}

\documentclass[a4paper,11pt]{article}
% \usepackage[]{natbib}
\usepackage[numbers]{natbib}
\bibliographystyle{plain-fr}
\usepackage{exptech}

\usepackage{lmodern}
\usepackage{amssymb,amsmath}
  \usepackage{textcomp} % provides euro and other symbols

\usepackage{fancyhdr}


\usepackage{xcolor}
\IfFileExists{xurl.sty}{\usepackage{xurl}}{} % add URL line breaks if available
\IfFileExists{bookmark.sty}{\usepackage{bookmark}}{\usepackage{hyperref}}
\hypersetup{
  pdftitle={Exercices cours de Secu sur les clouds},
  pdfauthor={Clément ; Romain ; Romain ; Maxime ; Volodia },
  pdfborder={0 0 0},
  breaklinks=true}
\urlstyle{same}  % don't use monospace font for urls




\usepackage{graphicx,grffile}
\makeatletter
\def\maxwidth{\ifdim\Gin@nat@width>\linewidth\linewidth\else\Gin@nat@width\fi}
\def\maxheight{\ifdim\Gin@nat@height>\textheight\textheight\else\Gin@nat@height\fi}
\makeatother
% Scale images if necessary, so that they will not overflow the page
% margins by default, and it is still possible to overwrite the defaults
% using explicit options in \includegraphics[width, height, ...]{}
\setkeys{Gin}{width=\maxwidth,height=\maxheight,keepaspectratio}

% Make links footnotes instead of hotlinks:
\DeclareRobustCommand{\href}[2]{#2\footnote{\url{#1}}}


\setlength{\emergencystretch}{3em}  % prevent overfull lines
\providecommand{\tightlist}{%
  \setlength{\itemsep}{0pt}\setlength{\parskip}{0pt}}

% 
% set default figure placement to htbp
\makeatletter
\def\fps@figure{htbp}
\makeatother


\title{\textbf{Exercices cours de Secu sur les clouds}}
\author{Clément \textsc{Chavanon} \and Romain \textsc{Hu} \and Romain
\textsc{Hubert} \and Maxime \textsc{Grimaud} \and Volodia
\textsc{Parol-Guarino} \and 
 \\ Encadrant : Pascal \textsc{Garcia}}

% \author{Francesco \textsc{Bariatti} \and Adrien \textsc{Gasté} \and Mikael \textsc{Le} \and Romain \textsc{Lebouc}
%         \\
%         Encadrant : Pascal \textsc{Garcia}}

\date{2019-2020}

\begin{document}
\maketitle
\begin{abstract}
Créer de toutes pièces une Intelligence Artificielle (IA) sur le jeu des
pingouins. Ce jeu est un jeu de stratégie sur plateau, sa principale
caractéristique vient de ses cases hexagonales. De plus ce jeu réagit
très bien lorsque soumis à une IA de type \emph{Monte-Carlo Tree
Search}, que nous avons codé. Le second défi de ce projet est également
sa plateforme cible : exécuter le code de l'interface et de l'IA dans un
navigateur moderne. Pour cela nous utilisons \texttt{Emscripten} qui
nous permet de compiler notre IA en \texttt{WebAssembly} et d'atteindre
des performances proches du natif. Quant au \emph{frontend}, c'est une
application classique \texttt{Angular}.

\end{abstract}

{
\tableofcontents
}
\hypertarget{introduction}{%
\section*{Introduction}\label{introduction}}
\addcontentsline{toc}{section}{Introduction}

\hypertarget{le-jeu-des-pingouins}{%
\subsection{Le jeu des pingouins}\label{le-jeu-des-pingouins}}

Le jeu des pingouins est un jeu de stratégie sur plateau de 4 joueurs.
Le plateau contient 60 cases hexagonales sur lesquelles se trouvent 1 à
3 poissons.

En début de partie, chaque joueur place un certain nombre de pingouins
(de 2 à 4 suivant le nombre de joueurs) sur le plateau. A chaque tour,
le joueur doit, si possible, bouger l'un de ses pingouins. Les
mouvements de ceux se font sur en ligne droite depuis les 6 faces de la
case hexagonale sur laquelle il se trouve. Il ne peut passer par des
trous ou au-dessus d'autres pingouins, peu importe leur couleur. Une
fois le mouvement achevé, la case de départ est retiré du plateau. Le
joueur peut alors incrémenter du nombre de poisson qu'il y avait sur
cette case son score.

Le jeu se termine lorsque plus aucun pingouins ne peut se déplacer. Le
joueur avec le plus de points (poissons) remporte la partie.

\hypertarget{notre-tuxe2che}{%
\subsection{Notre tâche}\label{notre-tuxe2che}}

\hypertarget{au-duxe9part}{%
\subsubsection{Au départ}\label{au-duxe9part}}

Le sujet portait sur l'implémentation de ce jeu dans un environnement
Web, en utilisant le nouveau standard \emph{WebAssembly}. Les sources du
projet sont compilés avec \emph{Emscripten} qui permet de coder en
\texttt{c++} pour la partie technique. L'interface devait se faire avec
les libraires \emph{Simple DirectMedia Layer}.

\hypertarget{bref-suivi}{%
\subsubsection{Bref suivi}\label{bref-suivi}}

Afin de tester la faisabilité et les différentes technologies, nous
avons décidé de procéder à la création de l'algorithme de façon
abstraite et de tester avec un jeu simple et facilement implémentable :
le morpion. Pour la partie graphique nous avions simplement codé en
JavaScript vanilla. En parallèle nous avons testé une autre technologie
pour cela : \texttt{PixiJS}. Cependant cela ne s'est pas avéré
satisfaisant pour notre utilisation et avons décidé de choisir quelque
chose de plus simple : \texttt{Angular}.

\hypertarget{nos-pruxe9duxe9cesseurs}{%
\subsubsection{Nos prédécesseurs}\label{nos-pruxe9duxe9cesseurs}}

Ce projet n'est pas nouveau. Une précédente équipe y a déjà passé de
nombreuses heures. Cependant, afin de simplifier notre travail il a été
décidé de tout refaire, y compris le MCTS dont le code leur avait été
donné déjà optimisé. En effet, notre technologie étant récente, le
\emph{multithreading} par exemple pouvait s'avérer plus compliqué à
porter en \emph{WebAssembly} qu'à réécrire.

\hypertarget{notre-objectif}{%
\subsubsection{Notre objectif}\label{notre-objectif}}

Principalement nous nous sommes concentré sur le fonctionnement correct
de tout le projet et pas seulement de l'algorithme et du jeu. C'est pour
cela que nous avons choisi de présenter un résultat plus correct
qu'optimal (par exemple nous n'avons pas utilisé de représentation en
\emph{bitboards}, comme l'on fait nos prédécesseurs, de même qu'ils
n'ont pas eu l'algorithme à gérer).

\hypertarget{ruxe9alisation}{%
\section{Réalisation}\label{ruxe9alisation}}

\hypertarget{repruxe9sentation-du-jeu}{%
\subsection{Représentation du jeu}\label{repruxe9sentation-du-jeu}}

\hypertarget{repruxe9sentation}{%
\subsubsection{Représentation}\label{repruxe9sentation}}

\hypertarget{mcts}{%
\subsection{MCTS}\label{mcts}}

\hypertarget{multihtreading}{%
\subsection{Multihtreading}\label{multihtreading}}

\hypertarget{interface-graphique}{%
\subsection{Interface graphique}\label{interface-graphique}}

Pour offrir une expérience de jeu optimale, et afin d'exporter le jeu
sur un navigateur, nous avons du mettre en place une interface graphique
pour notre jeu. Avec les contraintes de temps et les contraintes
techniques, nous avons été amenés à faire des choix aux niveaux des
technologies utilisées et des méthodes d'implémentation afin de pouvoir
produire rapidement une interface utilisable. \#\#\# Angular / Ionic
Afin de mettre en place, un code solide et rapidement exploitable, nous
voulions impérativement utiliser \texttt{Typescript}, pour mettre en
place le moteur de jeu côté graphisme. En effet, son contrôle de typage
est un véritable plus, par rapport à notre Proof Of Concept, où le
moteur du tic-tac-toe était en \emph{Javascript}. D'autre part, nous
voulions mettre en place une architecture de site Web plus globale qui
viendrait englober la partie véritablement jouable. Afin de mettre en
place cette architecture web sur pied au plus vite, nous nous sommes
tournés vers \texttt{Angular}.

Pour mettre en place la chartre graphique de notre application, nous
nous sommes tournés vers le framework \texttt{Ionic\ 4}, sorti
récemment, qui offre aux développeurs des thèmes pré-conçus et des
composants responsives. Basé sur \emph{Angular}, il s'intègre donc
parfaitement dans notre projet.

\hypertarget{organisation-de-lapplication}{%
\subsubsection{Organisation de
l'application}\label{organisation-de-lapplication}}

Dans sa version finale notre application se compose des pages
principales suivantes :

\begin{verbatim}
* une page d'accueil présentant le projet,
* une page avec le jeu en lui même,
* une page de présentation pour les membres de l'équipe,
* et une page pour les crédits.
\end{verbatim}

Cette dernière permet de présenter le projet dans sa globalité, ainsi
que les membres de l'équipe ayant participés à sa réalisation.

En utilisant \texttt{ngx\ rocket}, la base de l'application a pu être
générée rapidement et avec une qualité de production. De cette manière
notre application a pu disposé d'un service de routage et d'un autre de
traduction que nous avons agrémenté au fur et à mesure des différents
ajouts de pages et de fonctionnalités.

Durant nos recherches dans les différentes possibilités que pouvait nous
offrir \emph{Ionic}, nous avons mis en place la possibilité d'accéder à
une deuxième chartre graphique, définissant le \texttt{Dark\ Theme}.

\hypertarget{duxe9veloppement-du-jeu}{%
\subsubsection{Développement du jeu}\label{duxe9veloppement-du-jeu}}

\hypertarget{bindings-mcts}{%
\subsection{\texorpdfstring{\emph{Bindings}
MCTS}{Bindings MCTS}}\label{bindings-mcts}}

% % \bibliography{references}
% 
\bibliography{references}

\end{document}
