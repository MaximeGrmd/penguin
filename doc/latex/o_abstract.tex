Créer de toutes pièces une Intelligence Artificielle (IA) sur le jeu des
pingouins. Ce jeu est un jeu de stratégie sur plateau, sa principale
caractéristique vient de ses cases hexagonales. De plus ce jeu réagit
très bien lorsque soumis à une IA de type \emph{Monte-Carlo Tree
Search}, que nous avons codé. Le second défi de ce projet est également
sa plateforme cible : exécuter le code de l'interface et de l'IA dans un
navigateur moderne. Pour cela nous utilisons \emph{Emscripten} qui nous
permet de compiler notre IA en \texttt{WebAssembly} et d'atteindre des
performances proches du natif. Quant l'interface graphique, c'est une
application classique \texttt{Ionic} (un \emph{framework} basé sur
\emph{Angular}).
